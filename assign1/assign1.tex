\let\negmedspace\undefined
\let\negthickspace\undefined
\documentclass[journal,12pt,twocolumn]{IEEEtran}
\usepackage{cite}
\usepackage{amsmath,amssymb,amsfonts,amsthm}
\usepackage{algorithmic}
\usepackage{graphicx}
\usepackage{textcomp}
\usepackage{xcolor}
\usepackage{txfonts}
\usepackage{listings}
\usepackage{enumitem}
\usepackage{mathtools}
\usepackage{gensymb}
\usepackage{comment}
\usepackage[breaklinks=true]{hyperref}
\usepackage{tkz-euclide} 
\usepackage{listings}
\usepackage{gvv}                                        
\def\inputGnumericTable{}                                 
\usepackage[latin1]{inputenc}                                
\usepackage{color}                                            
\usepackage{array}                                            
\usepackage{longtable}                                       
\usepackage{calc}                                             
\usepackage{multirow}                                         
\usepackage{hhline}                                           
\usepackage{ifthen}                                           
\usepackage{lscape}

\newtheorem{theorem}{Theorem}[section]
\newtheorem{problem}{Problem}
\newtheorem{proposition}{Proposition}[section]
\newtheorem{lemma}{Lemma}[section]
\newtheorem{corollary}[theorem]{Corollary}
\newtheorem{example}{Example}[section]
\newtheorem{definition}[problem]{Definition}
\newcommand{\BEQA}{\begin{eqnarray}}
\newcommand{\EEQA}{\end{eqnarray}}
\newcommand{\define}{\stackrel{\triangle}{=}}
\theoremstyle{remark}
\newtheorem{rem}{Remark}
\begin{document}

\bibliographystyle{IEEEtran}
\vspace{3cm}

\title{10.5.4-5}
\author{EE23BTECH11033-killana jaswanth}
\maketitle
\newpage

\bigskip

\renewcommand{\thefigure}{\theenumi}
\renewcommand{\thetable}{\theenumi}
\textbf{Question}:\\
A small terrace at a football ground comprises of 15 steps each of which is 50
m long and built of solid concrete.Each step has a rise of 1/4 m and a tread of
1/2 m. Calculate the total volume of concrete required to build the terrace.
[Hint: Volume of concrete required to build the first step=\begin{align}
    \text{{volume}}=1/4 \cdot 1/2 \cdot 50 
\end{align}
\textbf{solution} 
\begin{align}
\text{{dimensions of any step}}= length\cdot breadth\cdot height
\end{align}
\\length of first step is l 
\\breadth of first step is b
\\height of first step is h.
\\
\\\begin{tabular}{|c|c| }
\hline
\textbf{parameter}& \textbf{value}
\\\hline
\multirow{3}{1em}\\\l&$50m$
\\b&$0.5m$
\\h&$0.25m$
\\\hline
\end{tabular}
\begin{align}
\text{{dimensions of first step}} = 50m \cdot 0.25m \cdot 0.5m
\end{align}
\\=volume of first step is 6.25$m^3$
\\=All the dimensions except height are same for all 15 steps .
\\=The height difference between any 2 consecutive steps is 0.25 m.
\\=so, the height of the second step is 0.25m+0.25m=0.5m
\\=So, the volume of the second step is 50m*5m*0.5m= 12.5$m^3$
\\=in the similar way the volume of the third step is 18.75$m^3$
\\=so, we can clearly notice that the volume of the steps are in arthimetic progression.
\\let the first term of the AP be $x(n)=6.25$
\\=the common difference is 6.25
\\=we have to find the sum of first 15 terms
\\=the formula of sum of first n terms in an AP is \begin{align}
S_n = \frac{n+1}{2} [2a+(n)d]   
\\ \textbf{here n starts from 0}
\end{align}
\\=n= number of terms
\\=a is  first term of the AP
\\d is the common difference
\\here\\
\\\begin{tabular}{|c|c| }
\hline
\textbf{parameter}& \textbf{value}
\\\hline
\multirow{3}{1em}\\a&$6.25$
\\d&$6.25$
\\n&$14$
\\\hline
\end{tabular}
\begin{align}
    S_n &= \frac{14+1}{2} [12.5+(14)6.25]
\end{align}
\begin{align}
S_n = \frac{15}{2}[12.5(14)6.25]
\end{align}
   \begin{align}
   S_n = \frac{15}{2} [12.5+87.5]
   \end{align}
   \begin{align}
   volume=(7.5) \cdot 100
   \end{align}
\\    volume is  750
\\hence, the volume of the total concerate is 750 $m^3$
\\\\\textbf{x(n) and u(n)}
\\let x(n) be the n'th term of the above AP
\\x(n)=a+nd
\\here 
\\ a is the first term of the AP
\\d is the common difference
\\n starts from zero
\\ \textbf{n'th term in terms of u(n)}
\\\begin{align}
    \text{{x(n)}}=(a+n \cdot d ) \cdot u(n)
\end{align}
\\\begin{tabular}{|c|c|c|c| }
\hline
\textbf{n}& \textbf{u(n)}& \textbf{x(n)}& \textbf{relation}
\\\hline
\multirow{3}{1em}\\$n<0$&0&0&x(n)=u(n)=0
\\0&1&a&x(n)=(a)(u(n))
\\$n>0$&1&(a+nd)&x(n)=(a+nd)u(n)
\\\hline
\end{tabular}


 \end{document}
