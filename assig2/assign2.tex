\let\negmedspace\undefined
\let\negthickspace\undefined
\documentclass[journal,12pt,twocolumn]{IEEEtran}
\usepackage{cite}
\usepackage{amsmath,amssymb,amsfonts,amsthm}
\usepackage{algorithmic}
\usepackage{graphicx}
\usepackage{textcomp}
\usepackage{xcolor}
\usepackage[justification=centering]{caption}
\usepackage{txfonts}
\usepackage{listings}
\usepackage{enumitem}
\usepackage{mathtools}
\usepackage{gensymb}
\usepackage{comment}
\usepackage[breaklinks=true]{hyperref}
\usepackage{tkz-euclide} 
\usepackage{listings}
\usepackage{gvv}                                        
\def\inputGnumericTable{}                                 
\usepackage[latin1]{inputenc}                                
\usepackage{color}                                            
\usepackage{array}                                            
\usepackage{longtable}                                       
\usepackage{calc}                                             
\usepackage{multirow}                                         
\usepackage{hhline}                                           
\usepackage{ifthen}                                           
\usepackage{lscape}

\newtheorem{theorem}{Theorem}[section]
\newtheorem{problem}{Problem}
\newtheorem{proposition}{Proposition}[section]
\newtheorem{lemma}{Lemma}[section]
\newtheorem{corollary}[theorem]{Corollary}
\newtheorem{example}{Example}[section]
\newtheorem{definition}[problem]{Definition}
\newcommand{\BEQA}{\begin{eqnarray}}
\newcommand{\EEQA}{\end{eqnarray}}
\newcommand{\define}{\stackrel{\triangle}{=}}
\theoremstyle{remark}
\newtheorem{rem}{Remark}
\begin{document}

\bibliographystyle{IEEEtran}
\vspace{3cm}

\title{11.9.5-13}
\author{EE23BTECH11033-killana jaswanth}
\maketitle
\newpage

\bigskip

\renewcommand{\thefigure}{\theenumi}
\renewcommand{\thetable}{\theenumi}
question:\begin{align}
\frac{a+bx}{a-bx}=\frac{b+cx}{b-cx}=\frac{c+dx}{c-dx}
\end{align}
then show that a,b,c,d are in G.P\\\\
solution:\\
\\\begin{table}[!ht]
 \centering
  \begin{tabular}{|c|c|c|}
\hline
\textbf{parameter}& \textbf{description}& \textbf{value}
\\\hline
\multirow{3}{1em}\\$x\brak{0}$&first term&$a$
\\\hline
$r$&common ratio&$\frac{x\brak{n}}{x\brak{n-1}}$
\\\hline
$n$&no of terms&$4$
\\\hline
$x\brak{n}$&\brak nth term&$x\brak{0}r^{n-1}$
\\\hline
\end{tabular}



   \caption{input parameters}
   \label{tab:11.9.5.13}
   \end{table}
\begin{align}
\frac{x\brak{0}+x\brak{1}x}{x\brak{0}-x\brak{1}x}&=\frac{x\brak{1}+x\brak{2}x}{x\brak{1}-x\brak{2}x}\\
x\brak{0}x\brak{3}x&=x\brak{1}^2x\\
\implies x\brak{1}^2&=x\brak{0}x\brak{2}
\end{align}
\begin{align}
\frac{x\brak{1}+x\brak{2}x}{x\brak{1}-x\brak{2}x}&=\frac{x\brak{2}+x\brak{3}x}{x\brak{2}-x\brak{3}x}\\
x\brak{1}x\brak{3}x&=x\brak{2}^2x\\
\implies x\brak{2}^2&=x\brak{1}x\brak{3}
\end{align}\\
x\brak{0},x\brak{1},x\brak{2} are in G.P and x\brak{1},x\brak{2},x\brak{3} are in G.P\\
So, x\brak{0},x\brak{1},x\brak{2},x\brak{3} are in G.P\\
Applying z-transform\\
\begin{align}
X\brak{z}&=\frac{x(0)}{1-rz^{-1}} \quad \abs{z}>\abs{r}\\
X\brak{z}&=\frac{a^2}{a-bz^{-1}}\quad \abs{z}>\abs{\frac{b}{a}}\\
X\brak{z}&=\frac{x\brak{0}^2}{x\brak{0}-x\brak{1}z^{-1}}\quad \abs{z}>\abs{\frac{x\brak{1}}{x\brak{0}}}
\end{align}
\end{document}
