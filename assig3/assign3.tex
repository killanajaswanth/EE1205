\let\negmedspace\undefined
\let\negthickspace\undefined
\documentclass[journal,12pt,twocolumn]{IEEEtran}
\usepackage{cite}
\usepackage{amsmath,amssymb,amsfonts,amsthm}
\usepackage{algorithmic}
\usepackage{graphicx}
\usepackage{textcomp}
\usepackage{xcolor}
\usepackage{txfonts}
\usepackage{listings}
\usepackage{enumitem}
\usepackage{mathtools}
\usepackage{gensymb}
\usepackage{comment}
\usepackage[breaklinks=true]{hyperref}
\usepackage{tkz-euclide} 
\usepackage{listings}
\usepackage{gvv}                                        
\def\inputGnumericTable{}                                 
\usepackage[latin1]{inputenc}                                
\usepackage{color}                                            
\usepackage{array}                                            
\usepackage{longtable}                                       
\usepackage{calc}                                             
\usepackage{multirow}                                         
\usepackage{hhline}                                           
\usepackage{ifthen}                                           
\usepackage{lscape}

\newtheorem{theorem}{Theorem}[section]
\newtheorem{problem}{Problem}
\newtheorem{proposition}{Proposition}[section]
\newtheorem{lemma}{Lemma}[section]
\newtheorem{corollary}[theorem]{Corollary}
\newtheorem{example}{Example}[section]
\newtheorem{definition}[problem]{Definition}
\newcommand{\BEQA}{\begin{eqnarray}}
\newcommand{\EEQA}{\end{eqnarray}}
\newcommand{\define}{\stackrel{\triangle}{=}}
\theoremstyle{remark}
\newtheorem{rem}{Remark}
\begin{document}

\bibliographystyle{IEEEtran}
\vspace{3cm}

\title{GATE-2023, EC-35}
\author{EE23BTECH11033- JASWANTH KILLANA}
\maketitle
\newpage
\bigskip

\renewcommand{\thefigure}{\theenumi}
\renewcommand{\thetable}{\theenumi}
\textbf{Question}:\\
In the circuit shown below, switch S was closed for a long time. If the switch is opened at t=0, the maximum magnitude of the voltage $V_R$ in volts is. (round off to nearest integer).\\
\textbf{solution} :
\begin{figure}[th]
\centering
\includegraphics[width=\linewidth]{/root/assign3/figs/gate.png}
\caption{}
\label{}
\end{figure}
\begin{align}
 At, t=0^-
\end{align}
inductor acts as wire\\
apply KVL in big loop
\begin{align}
-2+1i\brak{0^-}&=0\\
i\brak{0^-}&=2A
\end{align}
here\begin{table}[!ht]
 \centering
  \begin{tabular}{|c|c|c|}
\hline
\textbf{parameter}& \textbf{description}& \textbf{value}
\\\hline
\multirow{3}{1em}\\l&length&$50m$
\\\hline
b&breadth&$0.25m$
\\\hline
h&height&$0.5m$
\\\hline
V&volume&$6.25m^3$ 
\\\hline
\end{tabular}


   \caption{input parameters}
   \label{GATE-2023,EC-35}
   \end{table}
after t=0,\\
KVL, \begin{align}
2i\brak{t}+L\frac{di}{dt}
\end{align}
apply laplace transform,
 \begin{align}
2I\brak{s}-Li\brak{{0^{-}}}+LsI\brak{s}&=0\\
\implies I\brak{s}&=\frac{i\brak{{0^{-}}}}{s+2}\\
I\brak{s}&=\frac{2}{s+2}
 \end{align}
 applying inverse laplace transform
 \begin{align}
  I\brak{t}&=2\cdot e^{-2t}\\
  V_R&=-2I\brak{t}\\
  \implies V_R&=-4\cdot e^{-2t}
 \end{align}  
  As,
 \begin{align}
    t & \xrightarrow{} 0\\
     e^{-2t}&\xrightarrow{} 1\\
     \abs{V_R\brak{max}}&=4
 \end{align}
 \begin{figure}[th]
\centering
\includegraphics[width=\linewidth]{/root/assign3/figs/ilspice.png}
\caption{}
\label{}
\end{figure}
This is the simulation for Vr vs time\\
The magnitude of graph is max at $t=0$ of magnitude $4V$and then decreases with t. Which supports the solution by doing laplace transform method.
\end{document}

